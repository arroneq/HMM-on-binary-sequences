% !TeX program = lualatex
% !TeX encoding = utf8
% !TeX spellcheck = uk_Ua

\documentclass[12pt,mathserif]{beamer}
\usepackage{fontspec}
\setsansfont{Arial}
\usepackage[english, ukrainian]{babel}

\usetheme{Boadilla}
\usecolortheme{seahorse}

\makeatletter
\setbeamertemplate{footline}{
    \leavevmode%
    \hbox{%
        % here used "author -> title -> date" color beamer theme/font for fancy gradient looking
        % but hbox filled with \inserttitle -> \insertinstitute -> \insertframenumber
        \begin{beamercolorbox}[wd=0.3\paperwidth,ht=2.25ex,dp=1ex,center]{author in head/foot}%
            \usebeamerfont{author in head/foot}
            \insertshorttitle
        \end{beamercolorbox}%
        \begin{beamercolorbox}[wd=0.45\paperwidth,ht=2.25ex,dp=1ex,center]{title in head/foot}%
            \usebeamerfont{title in head/foot}
            \insertshortinstitute
        \end{beamercolorbox}%
        \begin{beamercolorbox}[wd=0.25\paperwidth,ht=2.25ex,dp=1ex,center]{date in head/foot}%
            \usebeamerfont{date in head/foot}
            Киїів \insertdate \hfill \insertframenumber{} / \inserttotalframenumber
        \end{beamercolorbox}%
    }%
    \vskip0pt%
}

% remove the navigation symbols
\makeatletter
\setbeamertemplate{navigation symbols}{}

\usepackage{tabularray}
\usepackage{mathtools,amsfonts}
\usepackage{bbm} % for using indicator \mathbbm{1} 
\usepackage{xurl}

\usepackage{tikz}
\usetikzlibrary{arrows.meta}
\usetikzlibrary{decorations.markings} % fancy arrows on a circle
\usetikzlibrary{bending} % for flexing and bending arrows
\usepackage{pgfplots}
\usepackage{pgfplotstable}
\pgfplotsset{compat=1.18}

\usepackage{float}
\usepackage{microtype}
\usepackage{cmap} % make LaTeX PDF output copy-and-pasteable
\usepackage{xcolor}

\DeclareMathOperator*{\argmin}{argmin}
\DeclareMathOperator*{\argmax}{argmax}
\newcommand*{\scaleq}[2][4]{\scalebox{#1}{$#2$}}

\usepackage{microtype}

\usepackage{amsthm}
\theoremstyle{plain}
\newtheorem{claim}{\indent Твердження}

\title{Звіт з переддипломної практики}
\author{Виконав студент 4 курсу групи ФІ-91 \\
спеціальності 113 Прикладна математика}
\institute[НТУУ <<КПІ ім. Ігоря Сікорського>> НН ФТІ ММАД]{НТУУ <<КПІ ім. Ігоря Сікорського>> НН ФТІ ММАД}
\date{2023}

\begin{document}

%%% --------------------------------------------------------------------
\section{Титульний стайд}
%%% --------------------------------------------------------------------

\begin{frame}
    \vspace{0.5cm}
    \begin{center}
        Звіт з переддипломної практики на тему
    \end{center}
    \begin{block}{}\centering\bfseries
        Програмна реалізація алгоритмів розв'язування задачі побудови оцінок параметрів частково спостережуваного ланцюга Маркова на бінарних послідовностях 
    \end{block}
    \vspace{1cm}
    \begin{columns}[t]
        \begin{column}{0.64\linewidth}
            \scriptsize
            \textbf{Тема дипломної роботи} \\
            \textit{Оцінювання параметрів частково \\ спостережуваного ланцюга Маркова \\ на бінарних послідовностях} \\ \vspace{2mm}
            \textbf{Тема дипломної роботи англійською} \\
            \textit{Parameter estimation of a partially observable \\ Markov chain on binary sequences}
        \end{column}
        \begin{column}{0.36\linewidth}
            \scriptsize
            \textbf{Виконав} \\
            \textit{студент 4 курсу, групи ФІ-91} \\
            \textit{113 <<Прикладна математика>>} \\
            \textit{Цибульник А. В.} \\ \vspace{2mm}
            \textbf{Науковий керівник} \\
            \textit{ст. викл. Наказной П. О.} \\ \vspace{2mm}
            \textbf{Консультант} \\
            \textit{к.ф.-\,м.н., доцент Ніщенко І. І.}
        \end{column}
    \end{columns}
\end{frame}

\begin{frame}
    \frametitle{Об'єкт, предмет та мета дослідження}
    \textbf{Об'єкт дослідження:} процеси, які описуються моделями частково спостережуваних ланцюгів Маркова.
    \vspace{0.5cm}

    \textbf{Предмет дослідження:} оцінки параметрів частково спостережуваного ланцюга Маркова на бінарних послідовностях.

    \vspace{0.5cm}
    \textbf{Мета дослідження:} розробка програмної реалізації алгоритмів розв'язування задачі побудови оцінок невідомих параметрів моделі з використанням математичного апарату прихованих марковських моделей та методів побудови статистичних оцінок.
\end{frame}

%%% --------------------------------------------------------------------
\section{Основні завдання дипломної роботи}
%%% --------------------------------------------------------------------

\begin{frame}
    \frametitle{\insertsection}
    \framesubtitle{(звіт з переддипломної практики)}
    \textbf{Задача навчання \textcolor{brown}{(виконано 100$\%$)}:}
    \begin{itemize}
        \item за наявними спостереженнями про динаміку набору функціоналів від станів прихованого ланцюга бінарних послідовностей оцінити керуючий параметр системи, використовуючи математичний апарат прихованих марковських моделей.
    \end{itemize}

    \vspace{0.5cm}
    \textbf{Задача декодування \textcolor{brown}{(виконано 100$\%$)}:}
    \begin{itemize}
        \item за наявними спостереженнями та оцінкою керуючого параметра відновити ланцюг прихованих станів.
    \end{itemize}
\end{frame}

\begin{frame}
    \frametitle{\insertsection}
    \framesubtitle{(звіт з переддипломної практики)}
    \textbf{Задача локалізації \textcolor{brown}{(виконано 100$\%$)}:}
    \begin{itemize}
        \item за відомими значеннями набору функціоналів від деякої невідомої підмножини стану прихованого ланцюга, оцінити потужність та набір елементів цієї підмножини.
    \end{itemize}

    \vspace{0.5cm}
    \textbf{Задача зашумленого навчання \textcolor{brown}{(виконано 100$\%$)}:}
    \begin{itemize}
        \item розв'язати задачу навчання, окреслену на попередньому слайді, враховуючи, що наявні спостереження є зашумленими, спотвореними.
    \end{itemize}
\end{frame}

\begin{frame}
    \frametitle{\insertsection}
    \framesubtitle{(продовження дипломної роботи)}
    \textbf{Задача навчання \textcolor{brown}{(виконано 100$\%$)}:}
    \begin{itemize}
        \item за наявними спостереженнями про динаміку набору функціоналів від станів прихованого ланцюга бінарних послідовностей оцінити керуючий параметр системи, використовуючи методи побудови статистичних оцінок.
    \end{itemize}
\end{frame}

%%% --------------------------------------------------------------------
\section{Стан формування звіту дипломної роботи}
%%% --------------------------------------------------------------------

\begin{frame}
    \frametitle{\insertsection}
    \begin{tblr}{
        % hlines,vlines,
        colspec={lcX},
        rows={m},
        column{2}={mode=math},
    }

    Вступ    & 100\%                 & повністю описаний \\
    Розділ 1 & 100\%                 & описаний повністю зі сформованими вис\-новками \\
    Розділ 2 & \textcolor{red}{50\%} & описані теоретичні викладки основних етапів дослідження \\
    Розділ 3 & \textcolor{red}{80\%} & описаний майже повністю та є частиною звіту з переддипломної практики \\
    Висновки & \textcolor{red}{50\%} & частково описані \\
    Додатки  & 100\%                 & повністю описані та є частиною звіту з переддипломної практики \\

    \end{tblr}
\end{frame}

%%% --------------------------------------------------------------------
\section{Апробація результатів та публікації}
%%% --------------------------------------------------------------------

\begin{frame}
    \frametitle{\insertsection}
    \begin{itemize}
        \item \textit{Цибульник А. В., Ніщенко І. І.} XXI Всеукраїнська науково-практична конференція студентів, аспірантів та молодих вчених <<Теоретичні i прикладні проблеми фізики, математики та інформатики>>.
        
        \vspace{2mm} Секція <<Математичне моделювання та аналіз даних>> (стр. 419\,-\,432).
        
        \vspace{2mm} 11-12 травня 2023 р., м. Київ.
    \end{itemize}
\end{frame}

\end{document}