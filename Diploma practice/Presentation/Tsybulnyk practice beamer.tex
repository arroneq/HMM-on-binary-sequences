% !TeX program = lualatex
% !TeX encoding = utf8
% !TeX spellcheck = uk_Ua

\documentclass[12pt,mathserif]{beamer}
\usepackage{fontspec}
\setsansfont{Arial}
\usepackage[english, ukrainian]{babel}

\usetheme{Boadilla}
\usecolortheme{seahorse}

\makeatletter
\setbeamertemplate{footline}{
    \leavevmode%
    \hbox{%
        % here used "author -> title -> date" color beamer theme/font for fancy gradient looking
        % but hbox filled with \inserttitle -> \insertinstitute -> \insertframenumber
        \begin{beamercolorbox}[wd=0.45\paperwidth,ht=2.25ex,dp=1ex,center]{author in head/foot}%
            \usebeamerfont{author in head/foot}
            \insertshorttitle
        \end{beamercolorbox}%
        \begin{beamercolorbox}[wd=0.45\paperwidth,ht=2.25ex,dp=1ex,center]{title in head/foot}%
            \usebeamerfont{title in head/foot}
            \insertshortinstitute
        \end{beamercolorbox}%
        \begin{beamercolorbox}[wd=0.1\paperwidth,ht=2.25ex,dp=1ex,center]{date in head/foot}%
            \usebeamerfont{date in head/foot}
            \insertframenumber{} / \inserttotalframenumber
        \end{beamercolorbox}%
    }%
    \vskip0pt%
}

% remove the navigation symbols
\makeatletter
\setbeamertemplate{navigation symbols}{}

\usepackage{tabularray}
\usepackage{mathtools,amsfonts}
\usepackage{bbm} % for using indicator \mathbbm{1} 
\usepackage{xurl}

\usepackage{tikz}
\usetikzlibrary{arrows.meta}
\usetikzlibrary{decorations.markings} % fancy arrows on a circle
\usetikzlibrary{bending} % for flexing and bending arrows
\usepackage{pgfplots}
\usepackage{pgfplotstable}
\pgfplotsset{compat=1.18}

\usepackage{float}
\usepackage{microtype}
\usepackage{cmap} % make LaTeX PDF output copy-and-pasteable
\usepackage{xcolor}

\DeclareMathOperator*{\argmin}{argmin}
\DeclareMathOperator*{\argmax}{argmax}
\newcommand*{\scaleq}[2][4]{\scalebox{#1}{$#2$}}

\usepackage{amsthm}
\theoremstyle{plain}
\newtheorem{claim}{\indent Твердження}

\title[Оцінювання характеристик ланцюга Маркова]{Оцінювання характеристик частково спостережуваного ланцюга Маркова на двійкових послідовностях}
\author[Цибульник, Ніщенко]{А.~В.~Цибульник \and І.~І.~Ніщенко}
\institute[Науково-практична конференція студентів]{Всеукраїнська науково-практична конференція студентів, аспірантів та молодих вчених}
\date{2023}

\begin{document}

\begin{frame}
    \titlepage
\end{frame}

\begin{frame}
    \frametitle{План доповіді}
    \tableofcontents
\end{frame}

\AtBeginSection[]
{
  \begin{frame}
    \frametitle{План доповіді}
    \tableofcontents[currentsection]
  \end{frame}
}

%%% --------------------------------------------------------------------
\section{Сфери застосування бінарних послідовностей}
%%% --------------------------------------------------------------------

\begin{frame}
    \frametitle{\insertsection} 
    \begin{itemize}
        \item Еволюція ДНК в біології \vspace{1cm}
        \item Бінарні послідовності в теорії інформації \vspace{1cm}
        \item Спінові системи у фізиці
    \end{itemize}
\end{frame}

%%% --------------------------------------------------------------------
\section{Моделювання об'єкту дослідження}
%%% --------------------------------------------------------------------

\begin{frame}
    \frametitle{\insertsection}
    Еволюцію бінарної послідовності довжини $N$ уявимо як випадкове блукання вершинами $N$-\,вимірного куба.  
    \begin{figure}[H]\centering
        \begin{tikzpicture}
    % mark apexes of a cube
    \coordinate[label=below left:{$(0,0,0)$}] (A) at (0,0,3);
    \coordinate[label=below left:{$(1,0,0)$}] (B) at (3,0,3);
    \coordinate[label=above right:{$(0,1,0)$}] (C) at (0,0,0);
    \coordinate[label=above right:{$(1,1,0)$}] (D) at (3,0,0);

    \coordinate[label=above left:{$(0,0,1)$}] (E) at (0,3,3);
    \coordinate[label=above left:{$(1,0,1)$}] (F) at (3,3,3); % [label={[shift={(-0.5,0)}]{$(1,0,1)$}}]
    \coordinate[label=above right:{$(0,1,1)$}] (G) at (0,3,0);
    \coordinate[label=above right:{$(1,1,1)$}] (H) at (3,3,0);

    % draw the cube (top & bottom)
    \draw[very thick] (C) -- (A) -- (B) -- (D) -- (C);
    \draw[very thick] (E) -- (G) -- (H) -- (F) -- (E);

    % draw the cube (front & back)
    \draw[very thick] (A) -- (E);
    \draw[very thick] (B) -- (F);
    \draw[very thick] (C) -- (G);
    \draw[very thick] (D) -- (H);

    % mark blue points in each apex
    \foreach \point in {A,B,C,D,E,F,G,H}{
        \fill[blue] (\point) circle [radius=2.5pt];
    }
\end{tikzpicture}
    \end{figure}
\end{frame}

\begin{frame}
    \frametitle{\insertsection}
    Ліниве блукання: з імовірністю $p$ залишаємося на місці, з імовірністю $\frac{1-p}{N}$ переходимо в сусідню вершину.
    \begin{columns}
        \begin{column}{0.6\linewidth}
            \begin{figure}[H]\centering
                \begin{tikzpicture}[font=\scriptsize, scale=0.8]
    % mark apexes of a cube
    \coordinate[label=below left:{$(0,0,0)$}] (A) at (0,0,3);
    \coordinate[label=below left:{$(1,0,0)$}] (B) at (3,0,3);
    \coordinate[label=above right:{$(0,1,0)$}] (C) at (0,0,0);
    \coordinate[label=above right:{$(1,1,0)$}] (D) at (3,0,0);

    \coordinate[label=above left:{$(0,0,1)$}] (E) at (0,3,3);
    \coordinate[label=above left:{$(1,0,1)$}] (F) at (3,3,3); % [label={[shift={(-0.5,0)}]{$(1,0,1)$}}]
    \coordinate[label=above right:{$(0,1,1)$}] (G) at (0,3,0);
    \coordinate[label=above right:{$(1,1,1)$}] (H) at (3,3,0);

    % rename some of them to create a path
    \coordinate[label={[red,font=\small]above left:{$x^1$}}] (G) at (0,3,0);
    \coordinate[label={[red,font=\small]above left:{$x^2$}}] (C) at (0,0,0);
    \coordinate[label={[red,font=\small]below right:{$x^3$}}] (D) at (3,0,0);
    \coordinate[label={[red,font=\small]below right:{$x^4$}}] (B) at (3,0,3);

    % draw the cube
    \draw[very thick] (C) -- (A) -- (B);
    \draw[very thick] (E) -- (G) -- (H) -- (F) -- (E);
    \draw[very thick] (E) -- (A);
    \draw[very thick] (D) -- (H);
    \draw[very thick] (F) -- (B);

    % mark blue points in each apex
    \foreach \point in {A,B,C,D,E,F,G,H}{
        \fill[blue] (\point) circle [radius=2.5pt];
    }

    % mark red points of the path
    \foreach \pathpoint in {G,C,D,B}{
        \fill[red] (\pathpoint) circle [radius=2.6pt];
    }

    % draw arrows in a path
    \draw[red,-{Stealth[scale=1.2]},shorten >= 3pt,line width=1pt] (G) -- (C);
    \draw[red,-{Stealth[scale=1.2]},shorten >= 3pt,line width=1pt] (C) -- (D);
    \draw[red,-{Stealth[scale=1.2]},shorten >= 3pt,line width=1pt] (D) -- (B);

    % some extentions:
    % 1) draw the text near an arrow
    % \draw[-{Stealth[scale=1.2]}, line width=1pt] (A) -- node [left] {$\frac{1-p}{3}$} +(0,2,0);
    % 2) draw circle-arrow near the point
    % \draw[
    %     -{Stealth[scale=1.2]},
    %     line width=1pt,
    % ] (0,0,3) arc (0:355:0.5) node[below right] {$p$};
\end{tikzpicture}
            \end{figure}
        \end{column}
        \begin{column}{0.4\linewidth}
            \begin{tblr}{
                    colspec={cc},
                    column{1,2}={mode=math},
                }
                    & x^t     \\
                t=1 & (0,1,1) \\
                t=2 & (0,1,0) \\
                t=3 & (1,1,0) \\
                t=4 & (1,0,0) \\
            \end{tblr}
        \end{column}
    \end{columns}
\end{frame}

\begin{frame}
    \frametitle{\insertsection}
    $\left\{ X^t \right\}_{t=\overline{1,T}}$ є ланцюгом Маркова зі станами в $E=\{0,1\}^N$ з початковим рівномірним розподілом $\pi:$
    \begin{equation*}
        \pi_{x}=P\left( X^{1}=x \right) = \frac{1}{2^N}
    \end{equation*}
    та матрицею перехідних імовірностей $A:$ 
    \begin{equation*}
        A_{xx'}=P\left( X^{t+1}=x'\,|\,X^{t}=x \right) = 
        \begin{cases*}
            p, & $d_H(x,x')=0$\\
            \tfrac{1-p}{N}, & $d_H(x,x')=1$ \\ 
            0, & інакше
        \end{cases*}
    \end{equation*}
\end{frame}

\begin{frame}
    \frametitle{\insertsection}
    Спостерігаємо набір функціоналів 
    \begin{equation*} 
        Y^t=(Y^t_1,\ldots,Y^t_L)=(\sum\limits_{i \in I_1}X^t_i,\ldots,\sum\limits_{i \in I_L}X^t_i),
    \end{equation*} 
    де $I_1,\ldots,I_L$ є заданими підмножинами $\{ 1,2,\ldots,N \}$.
    \vspace{0.5cm}

    \pause
    Наприклад, $N=12$, $I_1=(1,2,3)$, $I_2=(6,7,10,11,12)$
    \begin{table}\centering
        \begin{tblr}{
                colspec={ccc},
                column{1,2,3}={mode=math},
            }
                & x^t & y^t \\
            t=1 & \textcolor{orange!90!black}{010}01\textcolor{green6}{11}01\textcolor{green6}{101} 
                & (\textcolor{orange!90!black}{1},\textcolor{green6}{4}) \\
            t=2 & \textcolor{orange!90!black}{011}01\textcolor{green6}{11}01\textcolor{green6}{101} 
                & (\textcolor{orange!90!black}{2},\textcolor{green6}{4}) \\
            t=3 & \textcolor{orange!90!black}{011}01\textcolor{green6}{11}11\textcolor{green6}{101} 
                & (\textcolor{orange!90!black}{2},\textcolor{green6}{4}) \\
            t=4 & \textcolor{orange!90!black}{011}01\textcolor{green6}{11}11\textcolor{green6}{111} 
                & (\textcolor{orange!90!black}{2},\textcolor{green6}{5}) \\
        \end{tblr}
    \end{table}
\end{frame}

\begin{frame}
    \frametitle{\insertsection}
    \begin{claim}
        Послідовність $\left\{ \left( X^t,Y^t \right) \right\}_{t=\overline{1,T}}$ утворює приховану марковську модель $\left( \pi,A,B \right)$, де 
        \begin{equation*}
            B_{xy}=P\left( Y^t=y\,|\,X^t=x \right) = \prod\limits_{k=1}^{L} \mathbbm{1} \left( y_k=\sum\limits_{i \in I_k} x_i \right)
        \end{equation*} 
    \end{claim}
\end{frame}

%%% --------------------------------------------------------------------
\section{Побудова оцінок невідомих параметрів}
%%% --------------------------------------------------------------------

\begin{frame}
    \frametitle{Постановка задачі}
    \begin{enumerate}[1]
        \item Оцінити параметр $p$ за набором спостережень та декодувати послідовність станів прихованого ланцюга;
    \end{enumerate}
    \vspace{0.5cm}

    \pause
    Метод максимальної правдоподібності
    \begin{equation*}
        P\, \bigl( Y=y \,|\, p \bigr) = \sum_{x \in E^T} P\, \bigl( X=x,Y=y \,|\, p \bigr) \longrightarrow \max
    \end{equation*}

    Функція повної правдоподібності
    \begin{equation*}
        L_{p,x,y} = P\, \bigl( X=x,Y=y \,|\, p \bigr)
    \end{equation*}

    Відтак
    \begin{equation*}
        \widehat{\,p\,} = \argmax\limits_{p} \sum_{x \in E^T} L_{p,x,y}
    \end{equation*}
\end{frame}

\begin{frame}
    \frametitle{\insertsection}

    Ітераційний алгоритм Баума-Велша: 
    \begin{equation*}
        Q\left( p^{(n)},p \right) = \sum_{x \in E^T}L_{p^{(n)},x,y}\cdot\ln L_{p,x,y} \longrightarrow \max
    \end{equation*}

    Тож починаючи з деякого $p^{(0)}$
    \begin{equation*}
        p^{(n+1)} = \argmax\limits_{p} Q\left( p^{(n)},p \right)
    \end{equation*}
\end{frame}

\begin{frame}
    \frametitle{\insertsection}

    Формула переоцінки параметра $p:$
    \begin{equation*}
        p^{(n+1)} = p^{(n)}\cdot\frac{\sum\limits_{t=1}^{T-1}\sum\limits_{x \in E} \alpha_t(x)\,B_{xy^{t+1}}\,\beta_{t+1}(x)}{\sum\limits_{t=1}^{T-1}\sum\limits_{x \in E} \alpha_t(x)\,\beta_t(x)},
    \end{equation*}
    де 
    \begin{align*}
        & \alpha_t(x) = P\left( Y^1=y^1,\ldots,Y^t=y^t,\,X^t=x \,|\, p^{(n)} \right) \\
        & \beta_t(x) = P\left( Y^{t+1}=y^{t+1},\ldots,Y^T=y^T \,|\, X^t=x,\, p^{(n)} \right)
    \end{align*}
\end{frame}

\begin{frame}
    \frametitle{\insertsection}

    Алгоритм Вітербі: пошук такої послідовності прихованих станів $\widehat{X}^1,\widehat{X}^2,\ldots,\widehat{X}^T$, яка найкращим чином описує наявні спостереження:
    \begin{equation*}
        \widehat{X} = \argmax\limits_{x \in E^T} P\left( X=x\,|\,Y=y,\widehat{\,p\,} \right)
    \end{equation*}
\end{frame} 

\begin{frame}[t]
    \frametitle{Постановка задачі}
    \begin{enumerate}[2]
        \item Спостерігаємо значення $Y^t_{I_*}=\sum\limits_{i \in I_*}X^t_i$, де $I_*$~--- деяка невідома підмножина множини індексів.
    \end{enumerate}
    \vspace{0.25cm}

    \pause
    Наприклад, $N=12$, $I_1=(1,2,3)$, $I_2=(6,7,10,11,12)$, $I_*=?$
    \begin{center}
        \begin{tblr}{
            colspec={cccc},
            column{1-4}={mode=math},
        }
        & x^t & y^t & y^t_{I_*} \\
        t=1 & \textcolor{orange!90!black}{010}01\textcolor{green6}{11}01\textcolor{green6}{101} 
            & (\textcolor{orange!90!black}{1},\textcolor{green6}{4}) 
            & 3 \\
        t=2 & \textcolor{orange!90!black}{011}01\textcolor{green6}{11}01\textcolor{green6}{101} 
            & (\textcolor{orange!90!black}{2},\textcolor{green6}{4}) 
            & 3 \\
        t=3 & \textcolor{orange!90!black}{011}01\textcolor{green6}{11}11\textcolor{green6}{101} 
            & (\textcolor{orange!90!black}{2},\textcolor{green6}{4}) 
            & 4 \\
        t=4 & \textcolor{orange!90!black}{011}01\textcolor{green6}{11}11\textcolor{green6}{111} 
            & (\textcolor{orange!90!black}{2},\textcolor{green6}{5}) 
            & 4 \\
        \end{tblr}
    \end{center}  
    \vspace{0.25cm}

    \pause
    Яким чином можна відтворити елементи множини $I_*$ за спостереженнями $Y^1_{I_*},\ldots,Y^T_{I_*}$?;
\end{frame}

\begin{frame}
    \frametitle{\insertsection}\centering

    \begin{claim}
        Змістовною і незміщеною оцінкою потужності множини $I_*$ є статистика
        \begin{equation*}
            \widehat{|I_*|} = \frac{N}{1-p} \left( 1-\frac{1}{T-1}\sum_{t=1}^{T-1}\mathbbm{1}\left( Y^t_{I_*}=Y^{t+1}_{I_*} \right) \right) 
        \end{equation*}
    \end{claim}
\end{frame}

\begin{frame}
    \frametitle{\insertsection}

    Визначення компонент множини $I_*:$
    \begin{equation*}
        \widehat{I\,} = \argmin\limits_{1\leqslant k \leqslant C^{\widehat{|I_*|}}_N}{d\left( \widehat{Y\,}_{\mathtt{I_k}}, Y_{I_*} \right)},
    \end{equation*}

    тут 
    \begin{equation*}
        \widehat{Y\,}_{\mathtt{I_k}}=\sum\limits_{i \in \mathtt{I_k}}\widehat{X}^t_i
    \end{equation*}
    є сумою від декодованих елементів прихованого ланцюга.
    \vspace{0.8cm}

    \pause
    Що обрати в ролі міри близькості $d$?
\end{frame}

\begin{frame}
    \frametitle{\insertsection}

    Середньоквадратична відстань
    \begin{equation*}
        d_{S}\left( \widehat{Y\,}_{\mathtt{I_k}},Y_{I_*} \right) = \sum_{t=1}^{T}\left( \widehat{Y\,}^t_{\mathtt{I_k}} - Y^t_{I_*} \right)^2
    \end{equation*}

    Зважена відстань Жаккара
    \begin{equation*}
        d_{J}\left( \widehat{Y\,}_{\mathtt{I_k}},Y_{I_*} \right) = 1 - \frac{\sum\limits_{t=1}^{T}\min{\left( \widehat{Y\,}^t_{\mathtt{I_k}},Y^t_{I_*} \right)}}{\sum\limits_{t=1}^{T}\max{\left( \widehat{Y\,}^t_{\mathtt{I_k}},Y^t_{I_*} \right)}}
    \end{equation*}
\end{frame}

\begin{frame}[t]
    \frametitle{Постановка задачі}
    \begin{enumerate}[3]
        \item Спостереження на множинах $I_1,\ldots,I_L$ спотворюються ймовірностями $q_1,\ldots,q_L:$
    \end{enumerate}

    \begin{equation*}
        Y^t=\left( Y^t_k \right)_{k=\overline{1,L}} = \left( \sum_{i \in I_k} \widetilde{X}^t_i \right)_{k=\overline{1,L}}
    \end{equation*}
    де для $i \in I_k$
    \begin{equation*}
        \widetilde{X}^t_i =
        \begin{cases*}
            1 - X^t_i, & з імовірністю $q_k$ \\
            X^t_i, & з імовірністю $1 - q_k$ \\
        \end{cases*}
    \end{equation*}
    \vspace{0.25cm}

    Оцінити невідомий параметр моделі $p$ та ймовірності спотворень $q_1,q_2,\ldots,q_L$.
\end{frame}

\begin{frame}
    \frametitle{\insertsection}
    
    Наприклад, $N=12$, $I_1=(1,2,3)$, $I_2=(6,7,10,11,12)$

    \begin{center}
        \begin{tblr}{
            colspec={ccccc},
            column{1-5}={mode=math},
        }

            & x^t & \widetilde{x}^t & y^t & q \\
        t=1 & 010011101101 
            & 0\textcolor{red}{0}0011101101 
            & (0,4) 
            & (q_1,q_2) \\
        t=2 & 011011101101 
            & 01\textcolor{red}{0}011101101 
            & (1,4) 
            & (q_1,q_2) \\
        t=3 & 011011111101 
            & \textcolor{red}{1}110111111\textcolor{red}{1}1 
            & (3,5)
            & (q_1,q_2) \\
        t=4 & 011011111111 
            & 0110111111\textcolor{red}{00} 
            & (2,3) 
            & (q_1,q_2) \\
        
        \end{tblr}
    \end{center}
\end{frame}

\begin{frame}   
    \frametitle{\insertsection}

    \begin{claim}
        Якщо множини $I_1,\ldots,I_L$ є попарно неперетинними, то утворена послідовність $\left\{ \left( X^t,Y^t \right) \right\}_{t=\overline{1,T}}$ є прихованою марковською моделлю $\left( \pi,A,B^q \right)$, де 
        \begin{equation*}
            B^q_{xy} = P\left( Y^t=y\,|\,X^t=x \right) = \prod\limits_{k=1}^{L} P\left( \xi^k_{01}(x) + \xi^k_{11}(x) = y_k \right),
        \end{equation*}
        \begin{equation*}\scaleq[0.82]{
            \xi^k_{01}(x) \sim Bin\left( |I_k| - \sum\limits_{i \in I_k} x_i,\, q_k \right),\ \xi^k_{11}(x) \sim Bin\left( \sum\limits_{i \in I_k} x_i,\, 1 - q_k \right)},\ k=\overline{1,L}
        \end{equation*}
    \end{claim}
\end{frame}

\begin{frame}[t]
    \frametitle{\insertsection}

    Починаючи з деякого наближення моделі $\left( \pi,A^{(0)},B^{q^{(0)}} \right)$, формула переоцінки параметра $p:$
    \begin{equation*}
        p^{(n+1)} = p^{(n)}\cdot\frac{\sum\limits_{t=1}^{T-1}\sum\limits_{x \in E} \alpha_t(x)\,B^{q^{(n)}}_{xy^{t+1}}\,\beta_{t+1}(x)}{\sum\limits_{t=1}^{T-1}\sum\limits_{x \in E} \alpha_t(x)\,\beta_t(x)}
    \end{equation*}
    Формула переоцінки компонент вектора $q:$
    \begin{equation*}
        q_k^{(n+1)} = q_k^{(n)}\cdot\frac{\sum\limits_{t=1}^{T}\sum\limits_{x \in E}\beta_{t}(x)\sum\limits_{x' \in E} \alpha_{t-1}(x')\,A^{(n)}_{x'x}\sum\limits_{i \in I_k}P^{q^{(n)}}_{x,i}}{|I_k|\sum\limits_{t=1}^{T}\sum\limits_{x \in E} \alpha_t(x)\,\beta_t(x)}
    \end{equation*}
\end{frame}

%%% --------------------------------------------------------------------
\section{Результати чисельного експерименту}
%%% --------------------------------------------------------------------

\begin{frame}[t]
    \frametitle{\insertsection}
    Було згенеровано прихований ланцюг Маркова протягом $T=200$ моментів часу, $N=5$ та $p=0.2$. Множина спостережуваних індексів: $I=\{I_1,I_2\}=\{(2,3),(1,4)\}$.

    \begin{figure}[H]\centering
        \begin{tikzpicture}[scale=1.5]
    \node[style={circle, draw=black, very thick}] (x1) at (0,0) {$x_1$};
    \node[style={circle, draw=black, very thick}] (x2) at (2,0) {$x_2$};
    \node[style={circle, draw=black, very thick}] (x3) at (4,0) {$x_3$};
    \node[style={circle, draw=black, very thick}] (x4) at (0,-2) {$x_4$};
    \node[style={circle, draw=black, very thick}] (x5) at (2,-2) {$x_5$};

    % \draw[line width=1pt] (x1) -- (x4);
    % \draw[line width=1pt] (x2) -- (x3);
    \draw[green6,very thick] (3,0) ellipse (1.7cm and 0.55cm) 
        node[black,xshift=2.8cm,yshift=0.8cm] {$I_1$};
    \draw[orange!90!black,very thick] (0,-1) ellipse (0.55cm and 1.7cm)
        node[black,xshift=-1.5cm] {$I_2$};
\end{tikzpicture}
    \end{figure}
\end{frame}

\begin{frame}[t]
    \frametitle{\insertsection}
    \begin{enumerate}[1]
        \item Оцінити параметр $p$ за набором спостережень та декодувати послідовність станів прихованого ланцюга;
    \end{enumerate}

    \begin{columns}
        \begin{column}{0.62\linewidth}
            \begin{figure}[H]
                \begin{tikzpicture}[scale=0.8]
    \begin{axis}[
        xlabel = Ітерації алгоритму,
        ylabel = Значення параметра $\widehat{\,p\,}$,
        grid = both,
        ymax = 0.62,
        grid style = {draw=gray!30},
        minor tick num = 4,
        minor grid style = {draw=gray!10},
    ]
        \addplot[blue!80, mark=*] table {data/baum-welch learning algorithm.txt};
        \addplot[red, mark=*] table {
            n p
            0 0.55
        };
    \end{axis}
\end{tikzpicture}
            \end{figure}
        \end{column}
        \begin{column}{0.38\linewidth}
            \begin{tblr}{
                hlines,vlines,
                colspec={ccc},
                column{1-3}={mode=math},
            }

            p   & \widehat{\,p\,} & |p-\widehat{\,p\,}| \\
            0.2 & 0.1959          & 0.0041              \\

            \end{tblr}
        \end{column}
    \end{columns}
\end{frame}

\begin{frame}[t]
    \frametitle{\insertsection}
    \begin{enumerate}[1]
        \item Оцінити параметр $p$ за набором спостережень та декодувати послідовність станів прихованого ланцюга;
    \end{enumerate}

    \begin{columns}
        \begin{column}{0.62\linewidth}
            \begin{figure}[H]
                \begin{tikzpicture}
    \begin{axis}[
        symbolic x coords = {$d_H=0$,$d_H=1$,$d_H=2$,$d_H=3$,$d_H=4$,$d_H=5$},
        x tick label style = {font=\footnotesize},
        scale only axis,
        ymin = 0.0,
        xlabel = Значення відстані Геммінга,
        ylabel = Частка серед усіх станів,
        ymajorgrids = true,
        yminorgrids = true,
        grid style = {draw=gray!30},
        minor tick num = 3,
        xtick align = center,
        minor grid style = {draw=gray!10},
        minor x tick style = {draw=none},
    ]
        \addplot [
            ybar,
            bar width=25pt,
            fill=blue!80,
            opacity=0.7,
        ] coordinates {
            ($d_H=0$,0.17) ($d_H=1$,0.375) ($d_H=2$,0.135) ($d_H=3$,0.25) ($d_H=4$,0.045) ($d_H=5$,0.025)
        };
    \end{axis}
\end{tikzpicture}
            \end{figure}
        \end{column}
        \begin{column}{0.38\linewidth}
            \begin{equation*}\scaleq[0.8]{
                d_H\left( X^t,\widehat{X^t} \right) = \sum\limits_{i=1}^{N} \mathbbm{1}\left( X^t_i \neq \widehat{X^t_i} \right)
                }
            \end{equation*}
        \end{column}
    \end{columns}
\end{frame}

\begin{frame}[t]
    \frametitle{\insertsection}
    \begin{enumerate}[2]
        \item Відтворити елементи <<множини неявних індексів>> $I_*$;
    \end{enumerate}
    \vspace{0.5cm}

    Спостереження $I=\{I_1,I_2\}=\{(2,3),(1,4)\}$, неявні індекси покладемо $I_*=(1,3,5):$

    \begin{figure}[H]\centering
        \begin{tikzpicture}[scale=1.5]
        \node[style={circle, draw=black, fill=gray!20, very thick}] (x1) at (0,0) {$x_1$};
        \node[style={circle, draw=black, very thick}] (x2) at (2,0) {$x_2$};
        \node[style={circle, draw=black, fill=gray!20, very thick}] (x3) at (4,0) {$x_3$};
        \node[style={circle, draw=black, very thick}] (x4) at (0,-2) {$x_4$};
        \node[style={circle, draw=black, fill=gray!20, very thick}] (x5) at (2,-2) {$x_5$};;
        
        \draw[line width=1pt] (x1) -- (x4);
        \draw[line width=1pt] (x2) -- (x3);
\end{tikzpicture}
    \end{figure}
\end{frame}

\begin{frame}[t]
    \frametitle{\insertsection}
    \begin{enumerate}[2]
        \item Відтворити елементи <<множини неявних індексів>> $I_*$;
    \end{enumerate}
    \vspace{0.5cm}

    Залежність значення оцінки від довжини ланцюга
    \begin{table}
        \begin{tblr}{
            hlines,vlines,
            colspec={cccccc},
            row{1-3}={mode=math},
        }

        T               & 200    & 400    & 600    & 800    & 1000    \\
        \widehat{\,p\,} & 0.1959 & 0.1823 & 0.1882 & 0.2099 & 0.2092  \\
        \widehat{|I_*|} & 2      & 2      & 2      & 3      & 3       \\

        \end{tblr}
    \end{table}
    \vspace{0.5cm}

    \pause
    При малих $N$ можна використати оцінку
    \begin{equation*}
        \widehat{|I_*|}_{\max}=\max\limits_{1\leqslant t \leqslant T} Y^t_{I_*}
    \end{equation*}
\end{frame}

\begin{frame}[t]
    \frametitle{\insertsection}
    \begin{enumerate}[2]
        \item Відтворити елементи <<множини неявних індексів>> $I_*$;
    \end{enumerate}
    \vspace{0.5cm}

    Отримані результати:

    \begin{table}
        \begin{tblr}{
            hlines,vlines,
            colspec={lc},
            column{2}={mode=math},
        }

        Істинна множина $I_*$                                      & (1,3,5) \\
        Оцінка $\widehat{I\,}_S$ за середньоквадратичною відстанню & (1,2,5) \\
        Оцінка $\widehat{I\,}_J$ за зваженою відстанню Жаккара     & (1,2,3) \\

        \end{tblr}
    \end{table}
\end{frame}

\begin{frame}[t]
    \frametitle{\insertsection}
    \begin{enumerate}[3]
        \item Оцінити невідомий параметр моделі $p$ при ймовірностях спотворення $q_1,q_2,\ldots,q_L$.
    \end{enumerate}
    \vspace{0.25cm}

    Для $I_1,I_2$ було задано такі коефіцієнти спотворення:

    \begin{figure}[H]
        \begin{tikzpicture}[scale=1.5]
    \node[style={circle, draw=black, very thick}] (x1) at (0,0) {$x_1$};
    \node[style={circle, draw=black, very thick}] (x2) at (2,0) {$x_2$};
    \node[style={circle, draw=black, very thick}] (x3) at (4,0) {$x_3$};
    \node[style={circle, draw=black, very thick}] (x4) at (0,-2) {$x_4$};
    \node[style={circle, draw=black, very thick}] (x5) at (2,-2) {$x_5$};;

    \draw[green6,very thick] (3,0) ellipse (1.7cm and 0.55cm) 
        node[black,xshift=3.3cm,yshift=0.8cm] {$q_1=0.05$};
    \draw[orange!90!black,very thick] (0,-1) ellipse (0.55cm and 1.7cm)
        node[black,xshift=-2cm] {$q_2=0.1$};
\end{tikzpicture}
    \end{figure}
\end{frame}

\begin{frame}[t]
    \frametitle{\insertsection}
    \begin{enumerate}[3]
        \item Оцінити невідомий параметр моделі $p$ при ймовірностях спотворення $q_1,q_2,\ldots,q_L$.
    \end{enumerate}

    \begin{columns}
        \begin{column}{0.6\linewidth}
            \begin{figure}[H]
                \begin{tikzpicture}[scale=0.8]
    \begin{axis}[
        xlabel = Ітерації $n$ алгоритму,
        ylabel = Значення параметра $p^{(n)}$,
        ymax = 0.62,
        grid = both,
        grid style = {draw=gray!30},
        minor tick num = 4,
        minor grid style = {draw=gray!10},
    ]
        \addplot[blue!80, mark=*] table[x=n, y=p] {data/baum-welch distortions learning algorithm.txt};
        \addplot[red, mark=*] table[x=n, y=p] {
            n p
            0 0.55
        };
    \end{axis}
\end{tikzpicture}
            \end{figure}
        \end{column}
        \begin{column}{0.4\linewidth}
            \begin{tblr}{
                hlines,vlines,
                colspec={ccc},
                column{1-3}={mode=math},
            }

            p   & \widehat{\,p\,} & |p-\widehat{\,p\,}| \\
            0.2 & 0.2559          & 0.0559              \\

            \end{tblr}
        \end{column}
    \end{columns}
\end{frame}

\begin{frame}[t]
    \frametitle{\insertsection}
    \begin{enumerate}[3]
        \item Оцінити невідомий параметр моделі $p$ при ймовірностях спотворення $q_1,q_2,\ldots,q_L$.
    \end{enumerate}

    \begin{columns}
        \begin{column}{0.6\linewidth}
            \begin{figure}[H]
                \begin{tikzpicture}[scale=0.8]
    \begin{axis}[
        xlabel = Значення $\widehat{q\,}_1$,
        ylabel = Значення $\widehat{q\,}_2$,
        grid = both,
        grid style = {draw=gray!30},
        minor tick num = 4,
        minor grid style = {draw=gray!10},
        x tick label style={
            /pgf/number format/.cd,
            fixed,
            precision=2
        }
        % legend style={at={(0.05,0.95)}, anchor=north west, cells={anchor=west}, draw=none},
    ]
        \addplot[blue!80, mark=*] table[x=q1, y=q2] {data/baum-welch distortions learning algorithm.txt};
        \addplot[red, mark=*] table[x=q1, y=q2] {
            n p q1 q2
            0 0.55 0.3 0.4
        };
    \end{axis}
\end{tikzpicture}
            \end{figure}
        \end{column}
        \begin{column}{0.4\linewidth}
            \begin{tblr}{
                hlines,vlines,
                colspec={ccc},
                column{1-3}={mode=math},
            }

            q_1   & \widehat{q\,}_1 & |q_1-\widehat{q\,}_1| \\
            0.05  & 0.0454          & 0.0046                \\

            \end{tblr}
            
            \vspace{0.5cm}
            
            \begin{tblr}{
                hlines,vlines,
                colspec={ccc},
                column{1-3}={mode=math},
            }

            q_2     & \widehat{q\,}_2 & |q_2-\widehat{q\,}_2| \\
            0.1\,\  & 0.1184          & 0.0184                \\

            \end{tblr}
        \end{column}
    \end{columns}
\end{frame}

%%% --------------------------------------------------------------------
\section*{Висновки}
%%% --------------------------------------------------------------------

\begin{frame}
    \frametitle{\insertsection} 
    
    Невідомі параметри заданої моделі були оцінені 
    \begin{itemize}
        \item або шляхом побудови змістовних та незміщених статистичних оцінок;
        \item або за допомогою ітераційного алгоритму Баума-Велша.
    \end{itemize}
    \vspace{0.5cm}

    Результати чисельного експерименту продемонстрували ефективність використаних методів, зокрема збіжність побудованих оцінок до істинних значень параметрів при збільшенні кількості спостережень.
\end{frame}

\end{document}