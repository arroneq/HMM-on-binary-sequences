%!TEX root = ../thesis.tex
% створюємо вступ
Марковські моделі мають широкий та ефективний арсенал інструментів для аналізу динаміки систем, поведінка яких у кожен наступний момент часу зумовлюється лише поточним станом системи та не залежить від характеру еволюції у попередні моменти часу. 

Водночас у випадку, коли безпосереднє спостереження еволюції ланцюга Маркова є неможливим чи обмеженим, застосовують моделі прихованих ланцюгів Маркова (ПММ). У такому випадку аналіз поведінки процесу відбувається за деякою опосередкованою інформацією про <<приховані>>, справжні стани ланцюга. 

Наприклад, в біоінформатиці~\cite[глава 9]{Koski2001} апарат ланцюгів Маркова застосовують при дослідженні еволюції молекул ДНК протягом певного часу, вважаючи при цьому за стан системи зв'язану послідовність так званих нуклеотидів, які формуються над алфавітом чотирьох азотистих основ $\{\text{T, C, A, G} \}$.  

Існування статистичних залежностей в чергуванні фонем чи слів в природних мовах зумовлює ефективність використання прихованих марковських моделей до таких завдань, як створення голосових команди, служб транскрипції та голосових помічників~\cite{Rabiner1989}.

Не винятком стають і задачі розпізнавання мови жестів~\cite{Chaaraoui2013}: наприклад, представляючи жести як послідовності прихованих станів, ПММ можуть розпізнавати динаміку та варіації рухів рук.

Відтак, враховуючи актуальність вивчення еволюції систем, стани яких є послідовностями чи наборами символів певної довжини, у роботі розглядається ланцюг Маркова на множині двійкових послідовностей, динаміка якого відстежується за зміною в часі набору функціоналів від його станів.