%!TEX root = ../thesis.tex
% створюємо Висновки до всієї роботи
У першій частині звіту з переддипломної практики описані теоретичні викладки для розв'язування задачі побудови оцінок параметрів частково спостережуваного ланцюга Маркова на двійкових послідовностях. Невідомі параметри заданої моделі були оцінені або шляхом побудови змістовних та незміщених статистичних оцінок, або за допомогою ітераційного алгоритму Баума-Велша.

Друга частина звіту з переддипломної практики присвячена проведенню чисельного експерименту, результати якого продемонстрували ефективність використаних методів, зокрема збіжність побудованих оцінок до істинних значень параметрів при збільшенні кількості спостережень.

У рамках подальшого дослідження буде розглянута така постановка задачі навчання: за наявними спостереженнями про динаміку набору функціоналів від станів прихованого ланцюга бінарних послідовностей оцінити керуючий параметр системи, використовуючи методи побудови статистичних оцінок.