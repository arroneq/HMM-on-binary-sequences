% !TeX spellcheck = uk_Ua

У даному звіті викладена експериментальна перевірка ефективності використаних методів та алгоритмів для побудови оцінок невідомих параметрів частково спостережуваного ланцюга Маркова на бінарних послідовностях. А саме, продемонстровано результати таких задач:

\begin{enumerate}
    \item задача навчання: за наявними спостереженнями про динаміку набору функціоналів від станів прихованого ланцюга бінарних послідовностей оцінено керуючий параметр системи, використовуючи математичний апарат прихованих марковських моделей;
    \item задача декодування: за наявними спостереженнями та оцінкою керуючого параметра відновлено ланцюг прихованих станів;
    \item задачу локалізації: за відомими значеннями набору функціоналів від деякої невідомої підмножини стану прихованого ланцюга, оцінено потужність та набір елементів цієї підмножини;
    \item задача навчання, окреслена в пункті 1), враховуючи, що наявні спостереження є зашумленими, спотвореними.
\end{enumerate}

Для проведення чисельного експерименту, на основі якого виконуватиметься висновок щодо ефективності побудованих теоретичних оцінок невідомих параметрів, було розроблено відповідне програмне забезпечення.