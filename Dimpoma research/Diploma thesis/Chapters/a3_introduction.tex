%!TEX root = ../thesis.tex
% створюємо вступ
\textbf{Актуальність дослідження.} Марковські моделі мають широкий та ефективний арсенал інструментів для аналізу динаміки систем, поведінка яких у кожен наступний момент часу зумовлюється лише поточним станом системи та не залежить від характеру еволюції у попередні моменти часу. 

Водночас, у випадку, коли безпосереднє спостереження еволюції ланцюга Маркова є неможливим чи обмеженим, застосовують моделі прихованих ланцюгів Маркова (ПММ). У такому випадку аналіз поведінки процесу відбувається за деякою опосередкованою інформацією про <<приховані>>, справжні стани ланцюга. 

Наприклад, в біоінформатиці~\cite[глава 9]{Koski2001} апарат ланцюгів Маркова застосовують при дослідженні еволюції молекул ДНК протягом певного часу, вважаючи при цьому за стан системи зв'язану послідовність так званих нуклеотидів, які формуються над алфавітом чотирьох азотистих основ $\{\text{T, C, A, G} \}$.  

За рахунок використання ймовірнісних переходів між фонемами чи словами, приховані Марковські моделі є ефективним інструментом для таких завдань, як голосові команди, служби транскрипції та голосові помічники~\cite{Kumar2023}.

Не винятком стають і задачі розпізнавання мови жестів~\cite{Chaaraoui2013}: наприклад, представляючи жести як послідовності прихованих станів, ПММ можуть фіксувати динаміку та варіації рухів рук.

Відтак, враховуючи актуальність вивчення еволюції систем, стани яких є послідовностями, наборами символів певної довжини, у роботі розглядається дослідження динаміки бінарних послідовностей, вважаючи, що спостерігаються не безпосередньо самі послідовності, а лише деяка опосередкована, часткова інформація про них.

\textbf{Метою дослідження} є оцінка характеристик частково спостережуваної Марковської моделі на бінарних послідовностях. Для досягнення мети необхідно розв'язати \textbf{задачу дослідження}, яка полягає у вирішенні таких завдань:

\begin{enumerate}
\item провести огляд опублікованих джерел за тематикою дослідження;
\item перевірити, чи задана модель відповідає необхідним умовам для використання апарату прихованих Марковських ланцюгів;
\item побудувати оцінки згідно обраних методів;
\item експериментально перевірити ефективність отриманих теоретичних оцінок.
\end{enumerate}

\emph{Об'єктом дослідження} є процеси, які описуються моделями частково спостережуваних ланцюгів Маркова.

\emph{Предметом дослідження} є моделі та методи оцінки параметрiв частково спостережуваного ланцюга Маркова на бінарних послідовностях.

При розв’язанні поставлених завдань використовувались такі \emph{методи дослідження}: методи лінійної  алгебри, теорії імовірностей, математичної статистики, методи комп’ютерного та статистичного моделювання, методи та алгоритми дослідження ланцюгів Маркова.

\textbf{Наукова новизна} отриманих результатів полягає у \dots

\textbf{Практичне значення} результатів полягає у \dots

\textbf{Апробація результатів та публікації.} Частина даної роботи була представлена на XXI Науково-практичнiй конференцiї студентiв, аспiрантiв та молодих вчених «Теоретичнi i прикладнi проблеми фiзики, математики та iнформатики» (11-12 травня 2023 р., м. Київ).