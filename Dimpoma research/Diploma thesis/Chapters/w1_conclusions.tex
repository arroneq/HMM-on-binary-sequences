У роботі було розглянуто ланцюг Маркова на множині двійкових послідовностей, динаміка якого відстежується за зміною в часі набору функціоналів від його станів. З метою побудови оцінок невідомих параметрів заданої частково спостережуваної моделі було розв'язано усі поставлені завдання дослідження. 

Перш за все було перевірено, що досліджувана модель є прихованою марковською моделлю. Це дозволило використовувати у ході дослідження математичний апарат прихованих марковських моделей, а саме: ітераційний алгоритм Баума-Велша та алгоритм Вітербі. 

Наступним кроком за допомогою вищевказаних алгоритмів, а також із використанням методів математичної статистики за наявними спостереженнями про динаміку набору функціоналів від бінарних послідовностей було розв'язано такі задачі: побудовано оцінку керуючого параметра системи; відтворено послідовність прихованих станів; локалізовано джерело надходження значень набору функціоналів від деякої невідомої підмножини стану прихованого ланцюга; побудовано оцінку керуючого параметра системи, враховуючи зашумленість спостережуваних даних.

Наостанок експериментально перевірено ефективність використаних методів шляхом генерування об'єкта дослідження та виконання порівняльного аналізу обчислених оцінок із їхніми істинними значеннями.

Напрямки подальших досліджень можуть стосуватися розв'язків аналогічних задач для різних модифікацій розглянутої в роботі моделі. Наприклад, в ролі функціонала від станів прихованого ланцюга може виступати інша обчислювальна операція, відмінна від сумування. Також можна розглянути задачу оцінки керуючого параметра, вважаючи його неоднаковим для різних епох спостережень: для <<ранніх>> спостережень одне значення, для <<центральних>> інше, для <<прикінцевих>>~--- ще інше.