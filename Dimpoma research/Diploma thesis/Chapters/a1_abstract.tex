%!TEX root = ../abstract.tex

\abstractUkr

Кваліфікаційна робота містить: ??? стор., ??? рисунки, ??? таблиць, ??? джерел.

Об'єктом дослідження є ланцюг Маркова зі значеннями в множині двійкових послідовностей фіксованої довжини. Динаміка ланцюга задається як випадкове блукання вершинами одиничного куба, розмірність якого збігається з довжиною двійкової послідовності. 

Стани заданого ланцюга є неспостережуваними (прихованими). Спостережуваними величинами в кожен момент часу є набір значень певного функціонала від фіксованих підмножин двійкової послідовності, яка описує поточний стан прихованого ланцюга.

Метою дослідження є побудова оцінок невідомих параметрів заданої марковської моделі за допомогою математичного апарату прихованих марковських моделей та із використанням методів математичної статистики. 

Результати чисельного експерименту продемонстрували ефективність використаних методів, зокрема збіжність побудованих оцінок до істинних значень параметрів при збільшенні кількості спостережень.

% наприкінці анотації потрібно зазначити ключові слова
\MakeUppercase{ЛАНЦЮГ МАРКОВА, ПРИХОВАНА МАРКОВСЬКА МОДЕЛЬ, АЛГОРИТМ БАУМА-ВЕЛША, АЛГОРИТМ ВІТЕРБІ}

\abstractEng

Qualification work contains: ??? pages, ??? figures, ??? tables, ??? sources.

The object of the study is the Markov chain with values   in the set of binary sequences of fixed length. Chain dynamics is given as a random walk on the vertices of a unit cube whose dimension coincides with the length of the binary sequence.

The states of a given circuit are unobservable (hidden). The observed values at each time point are a set of values of a certain functional from fixed subsets of the binary sequence that describes the current state of the hidden chain.

The purpose of the study is to construct estimates of unknown parameters of a given Markov model using the mathematical apparatus of hidden Markov models and using methods of mathematical statistics.

The results of the numerical experiment demonstrated the effectiveness of the methods used, in particular the convergence of the constructed estimates to the true values of the parameters with an increase in the number of observations.

\MakeUppercase{MARKOV CHAIN, HIDDEN MARKOV MODEL, BAUM-WELCH ALGORITHM, VITERBI ALGORITHM}

% Не прибирайте даний рядок
\clearpage