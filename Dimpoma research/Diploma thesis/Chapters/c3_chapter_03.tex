%!TEX root = ../thesis.tex
\chapter{Результати чисельного експерименту}
\label{chap: practice}

\section{Моделювання об'єкта дослідження}

\section{Задача навчання}

\section{Задача декодування}

\section{Задача локалізації}

\section{Задача навчання за спотвореними спостереженнями}

% Подивіться, як нераціонально використовується простір, якщо не писати 
% вступи до розділів. :)

% Зазвичай третій розділ присвячено опису практичного застосування або 
% експериментальної перевірки аналітичних результатів, одержаних у другому 
% розділі роботи. Втім, це не обов'язкова вимога, і структура основної 
% частини диплому більш суттєво залежить від характеру поставлених завдань. 
% Навіть якщо у вас є певне експериментальне дослідження, але його загальний 
% опис займає дві сторінки, то краще приєднайте його підроздіром у 
% попередній розділ.

% При описі експериментальних досліджень необхідно:

% \begin{itemize}
% \item наводити повний опис експериментів, які проводились, параметрів 
% обчислювальних середовищ, засобів програмування тощо;
% \item наводити повний перелік одержаних результатів у чисельному вигляді для їх можливої 
% перевірки іншими особами;
% \item представляти одержані результати у вигляді таблиць та графіків, 
% зрозумілих людському оку;
% \item інтерпретувати одержані результати з точки зору поставленої задачі 
% та загальної проблематики ваших досліджень.
% \end{itemize}

% У жодному разі не потрібно вставляти у даний розділ тексти 
% інструментальних програм та засобів (окрім того рідкісного випадку, коли 
% саме тексти програм і є результатом проведення експериментів). За 
% необхідності тексти програм наводяться у додатках.

\chapconclude{\ref{chap: practice}}

Висновки до останнього розділу є, фактично, підсумковими під усім 
дослідженням; однак вони повинні стостуватись саме того, що розглядалось у 
розділі.