%!TEX root = ../thesis.tex
% створюємо розділ
\chapter{Оцінка параметрів моделі}
\label{chap: theory}

Структуруємо дослідження таким чином:

\begin{enumerate}
    \item спершу формально опишемо досліджуваний об'єкт, визначимо ключові параметри системи та переконаємося, що утворена модель є марковською;
    \item розв'яжемо задачу навчання: за наявними спостереженнями про динаміку набору функціоналів від станів прихованого ланцюга бінарних послідовностей оцінимо керуючий параметр системи, використовуючи математичий апарат прихованих марковських моделей та методи побудови статистичних оцінок;
    \item розв'яжемо задачу декодування: за наявними спостереженнями та оцінкою керуючого параметра відновимо ланцюг прихованих станів;
    \item розв'яжемо задачу локалізації: за відомими значеннями набору функціонів від деякої невідомої підмножини стану прихованого ланцюга, оцінимо потужність та набір елементів цієї підмножини;
    \item розв'яжемо задачу навчання, окреслену в пункті 2), враховуючи, що наявні спостереження є зашумленими, спотвореними.
\end{enumerate}

Кожен з підрозділів цього розділу матиме відповідну локанічну назву: <<Моделювання об'єкту дослідження>>, <<Задача навчання>>, <<Задача декодування>>, <<Задача локалізації>> та <<Задача навчання за спотвореними спостереженнями>>.

\section{Моделювання об'єкту дослідження}

% Розглянемо ланцюг Маркова $\left\{ X^t \right\}_{t=\overline{1,T}}$, який приймає значення зі скінченної множини $E=\{0,1\}^N$~--- множини всеможливих бінарних послідовностей довжини $N$.

% Динаміка ланцюга відбувається згідно узагальненої моделі Еренфестів: в кожен момент часу $t$ навмання обирається число $j$ з множини індексів $\left\{ 1,2,\ldots,N \right\}$ бінарної послідовності $X^t$ та відповідний елемент стану $X^t_j$ залишається незмінним з імовірністю $p$ або змінюється на протилежний бінарний символ з імовірністю $1-p$.

% Такого роду еволюцію бінарної послідовності довжини $N$ можна уявити як випадкове блукання вершинами $N$-\,вимірного куба. Наприклад, при довжині послідовностей $N=3$ випадкове блукання відбуватиметься вершинами такого куба: 

% \begin{figure}[H]\centering
%     \begin{minipage}[H]{0.6\linewidth}
%         \begin{figure}[H]\centering
%             \setfontsize{10pt}
%             \begin{tikzpicture}[font=\scriptsize, scale=0.8]
    % mark apexes of a cube
    \coordinate[label=below left:{$(0,0,0)$}] (A) at (0,0,3);
    \coordinate[label=below left:{$(1,0,0)$}] (B) at (3,0,3);
    \coordinate[label=above right:{$(0,1,0)$}] (C) at (0,0,0);
    \coordinate[label=above right:{$(1,1,0)$}] (D) at (3,0,0);

    \coordinate[label=above left:{$(0,0,1)$}] (E) at (0,3,3);
    \coordinate[label=above left:{$(1,0,1)$}] (F) at (3,3,3); % [label={[shift={(-0.5,0)}]{$(1,0,1)$}}]
    \coordinate[label=above right:{$(0,1,1)$}] (G) at (0,3,0);
    \coordinate[label=above right:{$(1,1,1)$}] (H) at (3,3,0);

    % rename some of them to create a path
    \coordinate[label={[red,font=\small]above left:{$x^1$}}] (G) at (0,3,0);
    \coordinate[label={[red,font=\small]above left:{$x^2$}}] (C) at (0,0,0);
    \coordinate[label={[red,font=\small]below right:{$x^3$}}] (D) at (3,0,0);
    \coordinate[label={[red,font=\small]below right:{$x^4$}}] (B) at (3,0,3);

    % draw the cube
    \draw[very thick] (C) -- (A) -- (B);
    \draw[very thick] (E) -- (G) -- (H) -- (F) -- (E);
    \draw[very thick] (E) -- (A);
    \draw[very thick] (D) -- (H);
    \draw[very thick] (F) -- (B);

    % mark blue points in each apex
    \foreach \point in {A,B,C,D,E,F,G,H}{
        \fill[blue] (\point) circle [radius=2.5pt];
    }

    % mark red points of the path
    \foreach \pathpoint in {G,C,D,B}{
        \fill[red] (\pathpoint) circle [radius=2.6pt];
    }

    % draw arrows in a path
    \draw[red,-{Stealth[scale=1.2]},shorten >= 3pt,line width=1pt] (G) -- (C);
    \draw[red,-{Stealth[scale=1.2]},shorten >= 3pt,line width=1pt] (C) -- (D);
    \draw[red,-{Stealth[scale=1.2]},shorten >= 3pt,line width=1pt] (D) -- (B);

    % some extentions:
    % 1) draw the text near an arrow
    % \draw[-{Stealth[scale=1.2]}, line width=1pt] (A) -- node [left] {$\frac{1-p}{3}$} +(0,2,0);
    % 2) draw circle-arrow near the point
    % \draw[
    %     -{Stealth[scale=1.2]},
    %     line width=1pt,
    % ] (0,0,3) arc (0:355:0.5) node[below right] {$p$};
\end{tikzpicture}
%         \end{figure}
%     \end{minipage}
%     \hfill
%     \begin{minipage}{0.35\linewidth}
%         \begin{table}[H]
%             \setfontsize{14pt}
%             \begin{tblr}{
%                     colspec={cc},
%                     column{1,2}={mode=math},
%                 }
%                     & x^t     \\
%                 t=1 & (0,1,1) \\
%                 t=2 & (0,1,0) \\
%                 t=3 & (1,1,0) \\
%                 t=4 & (1,0,0) \\
%             \end{tblr}
%         \end{table}
%     \end{minipage}
% \end{figure}

% Як наслідок окресленої динаміки, матриця перехідних імовірностей ланцюга матиме вигляд:
% \begin{equation*}\label{eq: A transition probabilities}
%     A_{xx'}=P\left( X^{t+1}=x'\,|\,X^{t}=x \right) = 
%     \scaleq[0.8]{
%     \begin{cases*}
%         p, & $x'=x$ \\
%         \dfrac{1-p}{N}, & 
%             $\begin{aligned} 
%                 & x^{'}_j = 1 - x_j \\ 
%                 & \forall i \neq j : x^{'}_i=x_i \\ 
%             \end{aligned}$ \\ 
%         0, & інакше
%     \end{cases*}}
% \end{equation*}

% Крім того, інваріантний розподіл $\pi=\left( \pi_x \right)_{x \in E}$ заданого ланцюга є рівномірним, тобто $\pi_x = \frac{1}{2^N}$. Вважатимемо, що початковий розподіл збігається з $\pi$.

% Наступним кроком введемо послідовність випадкових величин $\left\{ Y^t \right\}_{t=\overline{1,T}}$, які формуються таким чином: 
% \begin{equation}\label{eq: observations}
%     Y^t = \left( Y^t_k \right)_{k=\overline{1,L}} = \Bigl( \phi\left( X^t,I_k \right) \Bigr)_{k=\overline{1,L}},\ t=\overline{1,T},
% \end{equation}
% де $I=\left\{ I_1,\ldots,I_L \right\}$~--- задані підмножини множини індексів $\left\{ 1,2,\ldots,N \right\}$, а функціонал $\phi$ визначимо так:
% \begin{equation}\label{eq: phi function}
%     \phi\left( X^t,I_k \right) = \sum_{i \in I_k} X^t_i
% \end{equation}

% \begin{claim}
%     Послідовність $\left\{ \left( X^t,Y^t \right) \right\}_{t=\overline{1,T}}$ утворює приховану марковську модель $\left( \pi,A,B \right)$, де 
%     \begin{equation*}\label{eq: B emission probabilities}
%         B_{xy}=P\left( Y^t=y\,|\,X^t=x \right) = \prod\limits_{k=1}^{L} \mathbb{1}\left( y_k=\sum\limits_{i \in I_k} x_i \right)
%     \end{equation*}
% \end{claim}

\section{Задача навчання}

\section{Задача декодування}

\section{Задача локалізації}

\section{Задача навчання за спотвореними спостереженнями}

\chapconclude{\ref{chap: theory}}

Наприкінці розділу наводяться короткі підсумки.