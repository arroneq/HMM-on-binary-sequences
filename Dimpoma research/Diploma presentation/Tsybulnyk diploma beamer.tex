% !TeX program = lualatex
% !TeX encoding = utf8
% !TeX spellcheck = uk_Ua

\documentclass[12pt,mathserif]{beamer}
\usepackage{fontspec}
\setsansfont{Arial}
\usepackage[english, ukrainian]{babel}

\usetheme{Boadilla}
\usecolortheme{seahorse}

\makeatletter
\setbeamertemplate{footline}{
    \leavevmode%
    \hbox{%
        % here used "author -> title -> date" color beamer theme/font for fancy gradient looking
        % but hbox filled with \inserttitle -> \insertinstitute -> \insertframenumber
        \begin{beamercolorbox}[wd=0.65\paperwidth,ht=2.25ex,dp=1ex,center]{author in head/foot}%
            \usebeamerfont{author in head/foot}
            \insertshortinstitute
        \end{beamercolorbox}%
        \begin{beamercolorbox}[wd=0.25\paperwidth,ht=2.25ex,dp=1ex,center]{title in head/foot}%
            \usebeamerfont{title in head/foot}
            Київ\ \insertdate
        \end{beamercolorbox}%
        \begin{beamercolorbox}[wd=0.1\paperwidth,ht=2.25ex,dp=1ex,center]{date in head/foot}%
            \usebeamerfont{date in head/foot}
            \insertframenumber{} / \inserttotalframenumber
        \end{beamercolorbox}%
    }%
    \vskip0pt%
}

% remove the navigation symbols
\makeatletter
\setbeamertemplate{navigation symbols}{}

\usepackage{tabularray}
\usepackage{mathtools,amsfonts}
\usepackage{bbm} % for using indicator \mathbbm{1} 
\usepackage{xurl}

\usepackage{tikz}
\usetikzlibrary{arrows.meta}
\usepackage{pgfplots}
\usepackage{pgfplotstable}
\pgfplotsset{compat=1.18}

\usepackage{float}
\usepackage{microtype}
\usepackage{cmap} % make LaTeX PDF output copy-and-pasteable
\usepackage{xcolor}

\DeclareMathOperator*{\argmin}{argmin}
\DeclareMathOperator*{\argmax}{argmax}
\newcommand*{\scaleq}[2][4]{\scalebox{#1}{$#2$}}

\usepackage{amsthm}
\theoremstyle{plain}
\newtheorem{claim}{\indent Твердження}

\title{Оцінювання характеристик частково спостережуваного ланцюга Маркова на двійкових послідовностях}
\institute{НТУУ <<КПІ ім. Ігоря Сікорського>> НН ФТІ ММАД}
\date{2023}

\setbeamercolor{block title}{bg=blue!20, fg=black}

\begin{document}

\begin{frame}
    \begin{center}
        Дипломна робота на тему
    \end{center}
    \begin{block}{}
        \centering\Large
        \vspace{3mm}
        Оцінювання характеристик частково спостережуваного ланцюга Маркова на двійкових послідовностях
        \vspace{3mm} 
    \end{block}
    \begin{columns}[t]
        \begin{column}{0.6\linewidth}
        \end{column}
        \begin{column}{0.4\linewidth}
            \scriptsize
            \textbf{Виконав} \\
            \textit{студент 4 курсу, групи ФІ-91} \\
            \textit{113 <<Прикладна математика>>} \\
            \textit{Цибульник А. В.} \\ \vspace{2mm}
            \textbf{Науковий керівник} \\
            \textit{ст. викл. Наказной П. О.} \\ \vspace{2mm}
            \textbf{Консультант} \\
            \textit{к.ф.-\,м.н., доцент Ніщенко І. І.}
        \end{column}
    \end{columns}
\end{frame}

\begin{frame}
    \frametitle{Актуальність та мета дослідження}
    \textbf{Актуальність:} вивчення еволюції систем за частковою інформацією про динаміку їхніх станів, які, своєю чергою, є наборами символів певної довжини (ДНК, мова жестів).

    \vspace{1cm}
    \textbf{Мета дослідження:} за зміною в часі набору функціоналів від двійкових послідовностей побудувати оцінки невідомих параметрів моделі.
\end{frame}

\begin{frame}
    \frametitle{План доповіді}
    \tableofcontents
\end{frame}

% \AtBeginSection[]
% {
%   \begin{frame}
%     \frametitle{План доповіді}
%     \tableofcontents[currentsection]
%   \end{frame}
% }


%%% --------------------------------------------------------------------
\section{Моделювання об'єкта дослідження}
%%% --------------------------------------------------------------------

\begin{frame}
    \frametitle{\insertsection}
    Стан системи --- двійкова послідовність довжини $N$. 
    \vspace{0.5cm}

    Еволюція станів системи за узагальненою моделлю Еренфестів: навмання обраний символ стану $X^t$ з імовірністю $p$ не змінюється, а з імовірністю $(1-p)$~--- змінюється.
    \vspace{0.5cm}

    Наприклад, при $N=12$
    \begin{table}\centering
        \begin{tblr}{
                colspec={cc},
                column{1,2}={mode=math},
            }
                & x^t \\
            t=1 & 01\textbf{0}011011101            \\
            t=2 & 01\textbf{1}0110\textbf{1}1101   \\
            t=3 & \,0110110\textbf{0}11\textbf{0}1 \\
            t=4 & \,0110110011\textbf{0}1          \\
        \end{tblr}
    \end{table}
\end{frame}

\begin{frame}
    \frametitle{\insertsection}
    Спостерігаємо набір функціоналів 
    \begin{equation*} 
        Y^t = \left( Y^t_1,\ldots,Y^t_L \right) = \left( \sum\limits_{i \in I_1}X^t_i,\ldots,\sum\limits_{i \in I_L}X^t_i \right),
    \end{equation*} 
    де $I_1,\ldots,I_L$ є заданими підмножинами $\{ 1,2,\ldots,N \}$.
    \vspace{0.5cm}

    Наприклад, при $N=12$, $I_1=(1,2,3)$, $I_2=(6,7,10,11,12)$
    \begin{table}\centering
        \begin{tblr}{
                colspec={ccc},
                column{1,2,3}={mode=math},
            }
                & x^t & y^t \\
            t=1 & \textcolor{orange!90!black}{010}01\textcolor{green6}{10}11\textcolor{green6}{101} 
                & (\textcolor{orange!90!black}{1},\textcolor{green6}{3}) \\
            t=2 & \textcolor{orange!90!black}{011}01\textcolor{green6}{10}11\textcolor{green6}{101} 
                & (\textcolor{orange!90!black}{2},\textcolor{green6}{3}) \\
            t=3 & \textcolor{orange!90!black}{011}01\textcolor{green6}{10}01\textcolor{green6}{101} 
                & (\textcolor{orange!90!black}{2},\textcolor{green6}{3}) \\
            t=4 & \textcolor{orange!90!black}{011}01\textcolor{green6}{10}01\textcolor{green6}{101} 
                & (\textcolor{orange!90!black}{2},\textcolor{green6}{3}) \\
        \end{tblr}
    \end{table}
\end{frame}

% \begin{frame}
%     \frametitle{\insertsection}
%     \begin{claim}
%         Послідовність $\left\{ \left( X^t,Y^t \right) \right\}_{t=\overline{1,T}}$ утворює приховану марковську модель $\lambda = \left( \pi,A,B \right):$
%         \begin{equation*}
%             \pi_{x}=P\left( X^1=x \right) = \frac{1}{2^N}
%         \end{equation*}
%         \begin{equation*}
%             A_{xx'}=P\left( X^{t+1}=x'\,|\,X^{t}=x \right) = 
%             \begin{cases*}
%                 p, & $d_H(x,x')=0$\\
%                 \tfrac{1-p}{N}, & $d_H(x,x')=1$ \\ 
%                 0, & інакше
%             \end{cases*}
%         \end{equation*}
%         \begin{equation*}
%             B_{xy}=P\left( Y^t=y\,|\,X^t=x \right) = \prod\limits_{k=1}^{L} \mathbbm{1} \left( y_k=\sum\limits_{i \in I_k} x_i \right)
%         \end{equation*}
%     \end{claim}
% \end{frame}

%%% --------------------------------------------------------------------
\section{Побудова оцінок параметрів моделі}
%%% --------------------------------------------------------------------

\begin{frame}[t]
    \frametitle{\insertsection}
    \begin{enumerate}[1]
        \item За наявними частковими спостереженнями $\left\{ Y^t \right\}_{t=\overline{1,T}}$ про динаміку бінарних послідовностей оцінити керуючий параметр $p$ заданої марковської моделі.
    \end{enumerate}
    \vspace{0.5cm}

    Метод максимальної правдоподібності:
    \begin{equation*}
        \widehat{\,p\,} = \argmax\limits_{p} \sum_{x \in E^T} P\, \bigl( X=x,Y=y \,|\, p \bigr) \equiv \argmax\limits_{p} \sum_{x \in E^T} L_{p,x,y}
    \end{equation*}

    Ітераційний алгоритм Баума-Велша:
    \begin{equation*}
        p^{(n+1)} = \argmax\limits_{p} Q\left( p^{(n)},p \right) = \argmax\limits_{p} \sum_{x \in E^T}L_{p^{(n)},x,y} \ln L_{p,x,y}
    \end{equation*}
\end{frame}

\begin{frame}
    \frametitle{\insertsection}
    Формула переоцінки параметра $p$, починаючи з деякого $p^{(0)}:$
    \begin{equation*}
        p^{(n+1)} = p^{(n)}\cdot\frac{\sum\limits_{t=1}^{T-1}\sum\limits_{x \in E} \alpha_t(x)\,B_{xy^{t+1}}\,\beta_{t+1}(x)}{\sum\limits_{t=1}^{T-1}\sum\limits_{x \in E} \alpha_t(x)\,\beta_t(x)},
    \end{equation*}
    де 
    \begin{align*}
        & \alpha_t(x) = P\left( Y^1=y^1,\ldots,Y^t=y^t,\,X^t=x \,|\, p^{(n)} \right) \\
        & \beta_t(x) = P\left( Y^{t+1}=y^{t+1},\ldots,Y^T=y^T \,|\, X^t=x,\, p^{(n)} \right)
    \end{align*}
\end{frame}

\begin{frame}
    \frametitle{\insertsection}
    Крім того, побудовано змістовну та незміщену точкову оцінку параметра $p$ за допомогою методу моментів:
    \begin{equation*}
        \widehat{\,p\,} = 1-\frac{N}{\left| \bigcup\limits_{k=1}^{L}I_k \right|} \left( 1-\frac{1}{T-1}\sum_{t=1}^{T-1}\mathbbm{1} \Bigl( Y^t=Y^{t+1} \Bigr) \right)
    \end{equation*}
\end{frame}

\begin{frame}[t]
    \frametitle{\insertsection}
    \begin{enumerate}[2]
        \item За наявними спостереженнями $\left\{ Y^t \right\}_{t=\overline{1,T}}$ та оцінкою керуючого параметра $p^{(n)}$ відновити послідовність двійкових наборів.
    \end{enumerate}
    \vspace{0.5cm}

    Алгоритм Вітербі: пошук такої послідовності станів $\widehat{X\,}^1,\widehat{X\,}^2,\ldots,\widehat{X\,}^T$, яка найкращим чином описує наявні спостереження:
    \begin{equation*}
        \widehat{X\,} = \argmax\limits_{x \in E^T} P\left( X=x\,|\,Y=y,p^{(n)} \right)
    \end{equation*}
\end{frame} 

\begin{frame}[t]
    \frametitle{\insertsection}
    \begin{enumerate}[3]
        \item За відомими значеннями набору функціоналів від деякої невідомої підмножини $I_*$ стану прихованого ланцюга, оцінити потужність та набір елементів цієї підмножини.
    \end{enumerate}
    \vspace{0.5cm}
    
    Отже, спостерігаємо значення $Y^t_{I_*}=\sum\limits_{i \in I_*}X^t_i$.
    \vspace{0.25cm}

    Наприклад, $N=12$, $I_1=(1,2,3)$, $I_2=(6,7,10,11,12)$, $I_*=\,?$
    \begin{center}
        \begin{tblr}{
            colspec={cccc},
            column{1-4}={mode=math},
        }
        & x^t & y^t & y^t_{I_*} \\
        t=1 & \textcolor{orange!90!black}{010}01\textcolor{green6}{10}11\textcolor{green6}{101} 
            & (\textcolor{orange!90!black}{1},\textcolor{green6}{3}) 
            & 3 \\
        t=2 & \textcolor{orange!90!black}{011}01\textcolor{green6}{10}11\textcolor{green6}{101} 
            & (\textcolor{orange!90!black}{2},\textcolor{green6}{3}) 
            & 3 \\
        t=3 & \textcolor{orange!90!black}{011}01\textcolor{green6}{10}01\textcolor{green6}{101} 
            & (\textcolor{orange!90!black}{2},\textcolor{green6}{3}) 
            & 2 \\
        t=4 & \textcolor{orange!90!black}{011}01\textcolor{green6}{10}01\textcolor{green6}{101} 
            & (\textcolor{orange!90!black}{2},\textcolor{green6}{3}) 
            & 2 \\
        \end{tblr}
    \end{center}  
\end{frame}

\begin{frame}
    \frametitle{\insertsection}
    За набором спостережуваних <<сигналів>> $Y^1_{I_*},\ldots,Y^T_{I_*}$, оцінкою параметра $p^{(n)}$ та декодованим ланцюгом станів $\left\{ \widehat{X\,}^t \right\}_{t=\overline{1,T}}:$
    \vspace{0.5cm}

    \begin{itemize}
        \item побудовано змістовну та незміщену точкову оцінку потужності множини $I_*:$
        \begin{equation*}
            \widehat{|I_*|} = \frac{N}{1-p} \left( 1-\frac{1}{T-1}\sum_{t=1}^{T-1}\mathbbm{1}\Bigl( Y^t_{I_*}=Y^{t+1}_{I_*} \Bigr) \right) 
        \end{equation*}
        \item розроблено алгоритм визначення компонент множини $I_*.$
    \end{itemize}
\end{frame}

\begin{frame}[t]
    \frametitle{\insertsection}
    \begin{enumerate}[4]
        \item Спостереження на множинах $I_1,\ldots,I_L$ спотворюються з імовірностями $q_1,\ldots,q_L:$

    \begin{equation*}
        Y^t=\left( Y^t_k \right)_{k=\overline{1,L}} = \left( \sum_{i \in I_k} \widetilde{X\,}^t_i \right)_{k=\overline{1,L}}
    \end{equation*}
    де для $i \in I_k$
    \begin{equation*}
        \widetilde{X\,}^t_i =
        \begin{cases*}
            1 - X^t_i, & з імовірністю $q_k$ \\
            X^t_i, & з імовірністю $1 - q_k$ \\
        \end{cases*}
    \end{equation*}
\end{enumerate}

\end{frame}

\begin{frame}
    \frametitle{\insertsection}
    Наприклад, при $N=12$, $I_1=(1,2,3)$, $I_2=(6,7,10,11,12)$

    \begin{table}\centering
        \begin{tblr}{
            colspec={ccccc},
            column{1-5}={mode=math},
        }

            & x^t & \widetilde{x\,}^t & y^t & q \\
        t=1 & 010011011101 
            & 0\textcolor{red}{0}0011011101 
            & (0,3) 
            & (q_1,q_2) \\
        t=2 & 011011011101 
            & 01\textcolor{red}{0}011011101 
            & (1,3) 
            & (q_1,q_2) \\
        t=3 & 011011001101 
            & \textcolor{red}{1}110110011\textcolor{red}{1}1 
            & (3,4)
            & (q_1,q_2) \\
        t=4 & 011011001101 
            & 011011001\textcolor{red}{0}0\textcolor{red}{0}
            & (2,1) 
            & (q_1,q_2) \\
        
        \end{tblr}
    \end{table}
    \vspace{0.5cm}

    Задача: за спотвореними спостереженнями оцінити керуючий параметр моделі $p$ та вектор ймовірностей спотворення $q$, використовуючи ітераційний алгоритм Баума-Велша.
\end{frame}

% \begin{frame}   
%     \frametitle{\insertsection}

%     \begin{claim}
%         Якщо множини $I_1,\ldots,I_L$ є попарно неперетинними, то утворена послідовність $\left\{ \left( X^t,Y^t \right) \right\}_{t=\overline{1,T}}$ є прихованою марковською моделлю $\left( \pi,A,B^q \right)$, де 
%         \begin{equation*}
%             B^q_{xy} = P\left( Y^t=y\,|\,X^t=x \right) = \prod\limits_{k=1}^{L} P\left( \xi^k_{01}(x) + \xi^k_{11}(x) = y_k \right),
%         \end{equation*}
%         \begin{equation*}\scaleq[0.82]{
%             \xi^k_{01}(x) \sim Bin\left( |I_k| - \sum\limits_{i \in I_k} x_i,\, q_k \right),\ \xi^k_{11}(x) \sim Bin\left( \sum\limits_{i \in I_k} x_i,\, 1 - q_k \right)},\ k=\overline{1,L}
%         \end{equation*}
%     \end{claim}
% \end{frame}

\begin{frame}[t]
    \frametitle{\insertsection}
    Починаючи з деякого наближення $p^{(0)}$ та $q^{(0)}$, формула переоцінки параметра $p:$
    \begin{equation*}
        p^{(n+1)} = p^{(n)}\cdot\frac{\sum\limits_{t=1}^{T-1}\sum\limits_{x \in E} \alpha_t(x)\,B^{q^{(n)}}_{xy^{t+1}}\,\beta_{t+1}(x)}{\sum\limits_{t=1}^{T-1}\sum\limits_{x \in E} \alpha_t(x)\,\beta_t(x)}
    \end{equation*}
    Формула переоцінки компонент вектора $q:$
    \begin{equation*}
        q_k^{(n+1)} = q_k^{(n)}\cdot\frac{\sum\limits_{t=1}^{T}\sum\limits_{x \in E}\sum\limits_{x' \in E} \alpha_{t-1}(x')\,A^{(n)}_{x'x}\,\beta_{t}(x)\sum\limits_{i \in I_k}P^{q^{(n)}}_{x,i,1}}{|I_k|\sum\limits_{t=1}^{T}\sum\limits_{x \in E} \alpha_t(x)\,\beta_t(x)}
    \end{equation*}
\end{frame}

%%% --------------------------------------------------------------------
\section{Результати чисельного експерименту}
%%% --------------------------------------------------------------------

\begin{frame}[t]
    \frametitle{\insertsection}
    Було згенеровано прихований ланцюг Маркова протягом $T=200$ моментів часу, $N=5$ та $p=0.2$. Множина спостережуваних індексів $I=\{I_1,I_2\}=\{(2,3),(1,4)\}:$

    \begin{figure}[H]\centering
        \begin{tikzpicture}[scale=1.5]
        \node[style={circle, draw=black, very thick}] (x1) at (0,0) {$x_1$};
        \node[style={circle, draw=black, very thick}] (x2) at (2,0) {$x_2$};
        \node[style={circle, draw=black, very thick}] (x3) at (4,0) {$x_3$};
        \node[style={circle, draw=black, very thick}] (x4) at (0,-2) {$x_4$};
        \node[style={circle, draw=black, very thick}] (x5) at (2,-2) {$x_5$};
        
        \draw[line width=1pt] (x1) -- (x4);
        \draw[line width=1pt] (x2) -- (x3);
\end{tikzpicture}
    \end{figure}
\end{frame}

\begin{frame}[t]
    \frametitle{\insertsection}
    \begin{itemize}
        \item За наявними частковими спостереженнями про динаміку бінарних послідовностей оцінено керуючий параметр $p:$
    \end{itemize}
    
    \begin{columns}
        \begin{column}{0.6\linewidth}
            \begin{figure}[H]
                \begin{tikzpicture}[scale=0.8]
    \begin{axis}[
        xlabel = Ітерації $n$ алгоритму,
        ylabel = Значення параметра $p^{(n)}$,
        grid = both,
        ymax = 0.62,
        grid style = {draw=gray!30},
        minor tick num = 4,
        minor grid style = {draw=gray!10},
    ]
        \addplot[blue!80, mark=*] table {data/baum-welch learning algorithm.txt};
        \addplot[red, mark=*] table {
            n p
            0 0.55
        };
    \end{axis}
\end{tikzpicture}
            \end{figure}
        \end{column}
        \begin{column}{0.4\linewidth}
            \begin{tblr}{
                hlines,vlines,
                colspec={ccc},
                column{1-3}={mode=math},
            }

            p   & p^{(12)} & |p- p^{(12)}| \\
            0.2 & 0.1959   & 0.0041        \\

            \end{tblr}
        \end{column}
    \end{columns}
\end{frame}

\begin{frame}[t]
    \frametitle{\insertsection}
    Для спостережуваних множин $I_1,I_2$ згенерованого ланцюга було задано такі коефіцієнти спотворення:
    \vspace{0.3cm}

    \begin{figure}[H]
        \begin{tikzpicture}[scale=1.5]
        \node[style={circle, draw=black, very thick}] (x1) at (0,0) {$x_1$};
        \node[style={circle, draw=black, very thick}] (x2) at (2,0) {$x_2$};
        \node[style={circle, draw=black, very thick}] (x3) at (4,0) {$x_3$};
        \node[style={circle, draw=black, very thick}] (x4) at (0,-2) {$x_4$};
        \node[style={circle, draw=black, very thick}] (x5) at (2,-2) {$x_5$};;
        
        \draw[line width=1pt] (x1) -- node [left=0.3cm] {$q_2=0.1$} (x4);
        \draw[line width=1pt] (x2) -- node [above=0.25cm] {$q_1=0.05$} (x3);
\end{tikzpicture}
    \end{figure}
\end{frame}

\begin{frame}[t]
    \frametitle{\insertsection}
    \begin{itemize}
        \item За спотвореними спостереженнями оцінено керуючий параметр моделі $p$ та вектор ймовірностей $q:$
    \end{itemize}

    \begin{columns}
        \begin{column}{0.6\linewidth}
            \begin{figure}[H]
                \begin{tikzpicture}[scale=0.8]
    \begin{axis}[
        xlabel={$n$\,-\,та ітерація алгоритму},
        ylabel={Значення параметра $p^{(n)}$},
        ymax=0.62, ymin=0.18,
        grid=both,
        grid style={draw=gray!30},
        minor grid style={draw=gray!10},
        minor x tick num=4,
        minor y tick num=3,
    ]
        \addplot[blue!80, mark=*] table[x=n, y=p] {Data/baum-welch distortion learning algorithm.txt};
        \addplot[red, mark=*] table[x=n, y=p] {
            n p
            0 0.55
        };
    \end{axis}
\end{tikzpicture}
            \end{figure}
        \end{column}
        \begin{column}{0.45\linewidth}
            \begin{tblr}{
                hlines,vlines,
                colspec={ccc},
                column{1-3}={mode=math},
            }

            p   & p^{(53)} & |p-p^{(53)}| \\
            0.2 & 0.2559   & 0.0559       \\

            \end{tblr}
        \end{column}
    \end{columns}
\end{frame}

\begin{frame}[t]
    \frametitle{\insertsection}
    \begin{itemize}
        \item За спотвореними спостереженнями оцінено керуючий параметр моделі $p$ та вектор ймовірностей $q:$
    \end{itemize}

    \begin{columns}
        \begin{column}{0.6\linewidth}
            \begin{figure}[H]
                \begin{tikzpicture}
    \begin{axis}[
        xlabel={Значення $q_1^{(n)}$},
        ylabel={Значення $q_2^{(n)}$},
        scale only axis,
        grid=both,
        grid style={draw=gray!30},
        minor tick num=4,
        minor grid style={draw=gray!10},
        x tick label style={
            /pgf/number format/.cd,
            fixed,
            precision=2
        }
    ]
        \addplot[blue!80, mark=*] table[x=q1, y=q2] {Data/p & q distortion estimation.txt};
        \addplot[red, mark=*] table[x=q1, y=q2] {
            n p q1 q2
            0 0.55 0.3 0.4
        };
    \end{axis}
\end{tikzpicture}
            \end{figure}
        \end{column}
        \begin{column}{0.45\linewidth}
            \begin{tblr}{
                hlines,vlines,
                colspec={ccc},
                column{1}={wd=0.8cm},
                column{1-3}={mode=math},
            }

            q_1   & q_1^{(53)} & |q_1-q_1^{(53)}| \\
            0.05  & 0.0454     & 0.0046           \\

            \end{tblr}
            
            \vspace{0.5cm}
            
            \begin{tblr}{
                hlines,vlines,
                colspec={ccc},
                column{1}={wd=0.8cm},
                column{1-3}={mode=math},
            }

            q_2  & q_2^{(53)} & |q_2-q_2^{(53)}| \\
            0.1  & 0.1184     & 0.0184         \\

            \end{tblr}
        \end{column}
    \end{columns}
\end{frame}

%%% --------------------------------------------------------------------
\section*{Висновки}
%%% --------------------------------------------------------------------

\begin{frame}
    \frametitle{\insertsection} 
    Невідомі параметри заданої моделі були оцінені 
    \begin{itemize}
        \item або шляхом побудови змістовних та незміщених оцінок за допомогою методу моментів;
        \item або за допомогою ітераційного алгоритму Баума-Велша.
    \end{itemize}
    \vspace{0.5cm}

    Результати чисельного експерименту продемонстрували ефективність використаних методів, зокрема збіжність побудованих оцінок до істинних значень параметрів при збільшенні кількості спостережень.
\end{frame}

%%% --------------------------------------------------------------------
\section*{Апробація результатів та публікації}
%%% --------------------------------------------------------------------

\begin{frame}
    \frametitle{\insertsection}
    \begin{itemize}
        \item \textit{Цибульник А. В., Ніщенко І. І.,} XXI Всеукраїнська науково-практична конференція студентів, аспірантів та молодих вчених <<Теоретичні i прикладні проблеми фізики, математики та інформатики>>.
        
        \vspace{4mm} Секція <<Математичне моделювання та аналіз даних>> (стр. 419\,--\,432).
        
        \vspace{4mm} 11-12 травня 2023 р., м. Київ.
    \end{itemize}
\end{frame}

\end{document}